\chapter{Lineare Algebra}
\section{Matrixinverse}
\subsection{$1 \times 1$}
\begin{equation}
    {\begin{pmatrix}
        a_{11}
    \end{pmatrix}}^{-1} =
    \begin{pmatrix}
        \frac{1}{a_{11}}
    \end{pmatrix} 
\end{equation}

\subsection{$2 \times 2$}
\begin{equation}
    {\begin{pmatrix}
        a_{11} & a_{12} \\
        a_{21} & a_{22}
    \end{pmatrix}}^{-1} =
    \frac{1}{\det A}
    \begin{pmatrix}
        a_{22} & -a_{12} \\
        -a_{21} & a_{11}
    \end{pmatrix} 
\end{equation}

\subsection{Höhere Ordnung}
\subsubsection{Gauß-Verfahren}
Gleichungssystem:
\begin{equation}
    A \cdot A^{-1} = I
\end{equation}
mit Gaussverfahren lösen.

\subsubsection{Über Adjunkte}
Ansatz:
\begin{equation}
    A^{-1} = \frac{\text{adj} (A)}{\det (A)}
\end{equation}

\paragraph{Adjunkte bestimmen}
\begin{enumerate}
    \item
        Zu jedem Eintrag den Minor berechnen:

        Zeile und Spalte des Elements streichen und Determinante der Rest-Matrix.

    \item
        Das Vorzeichen jedes Minors wird durch ${(-1)}^{i+j}$ bestimmt

    \item
        Die Adjunkte ist die transponierte Matrix der Vorzeichenbehafteten Adjunkten
\end{enumerate}

\section{Determinante}
\subsection{$1 \times 1$}
\begin{equation}
    \det \begin{pmatrix} a_{11} \end{pmatrix} = a_{11}
\end{equation}

\subsection{$2 \times 2$}
\begin{equation}
    \det
        \begin{pmatrix}
            a_{11} & a_{12} \\
            a_{21} & a_{22}
        \end{pmatrix}
    = a_{11} \cdot a_{22} - a_{12} \cdot a_{21}
\end{equation}

\subsection{$3 \times 3$}
\begin{equation}
    \det
        \begin{pmatrix}
            a_{11} & a_{12} & a_{13}\\
            a_{21} & a_{22} & a_{23} \\
            a_{31} & a_{32} & a_{33}
        \end{pmatrix}
    = a_{11} a_{22} a_{33} + a_{12} a_{23} a_{21} + a_{13} a_{21} a_{22} -
    a_{31} a_{22} a_{13} - a_{32} a_{23} a_{11} - a_{33} a_{21} a_{12}
\end{equation}

\subsection{Höhere Ordnung}
\subsubsection{Entwicklungssatz von Laplace}
Sei $A \in \mathbb{R}^{n \times n}$ und bezeichne $A_{ij}$ die $(n-1)\times(n-1)$ Matrix, die aus $A$ durch Streichen
der $i$-ten Zeile und $j$-ten Spalte entsteht. Die Einträge von $A$ bezeichnen wir mit $A = (a_{ij})_{1 \leq i,j \leq n}$. Dann gilt:
\begin{enumerate}[label= (\alph*)]
    \item Für $j \in \{1, \ldots, n \}$ beliebig aber fest ist
        \begin{equation*}
            \det(A) = \sum_{i=1}^n a_{ij} \det{A_{ij}} {(-1)}^{i+j}
        \end{equation*}
    \item Für $i \in \{1, \ldots, n \}$ beliebig aber fest ist
        \begin{equation*}
            \det(A) = \sum_{j=1}^n a_{ij} \det{A_{ij}} {(-1)}^{i+j}
        \end{equation*}
\end{enumerate}

\subsubsection{Leibniz-Formel für Determinanten}
Sei $A \in \mathbb{R}^{n \times n}$ und bezeichne $S_n$ die Gruppe der Permutationen von
$\{1, \ldots, n\}$ und sei $\text{sgn}(\sigma)$ definiert durch $\text{sgn}: S_n \to \{-1, 1\}$:
\begin{equation*}
    \text{sgn}{\sigma} = \begin{cases}
        1 & \text{ falls $\sigma$ durch eine gerade Anzahl Permutationen entsteht} \\
        -1 & \text{ sonst}
    \end{cases}
\end{equation*}
wobei eine Transposition einer paarweisen Vertauschung entspricht.

Dann gilt:
\begin{equation*}
    \det(A) = \det((a_{ij})_{1\leq  i,j \leq n}) = 
    \sum_{\sigma \in S_n} \text{sgn}(\sigma) \prod_{k=1}^n a_{k\sigma(k)}
\end{equation*}
