\chapter{Flachheitsbasierte Folgeregelung}
\section{Definition flacher Systeme}
Ein nichtlineares Mehrgrößensystem $n$-ter Ordnung:
\begin{equation}
    \dot{x} = f(x,u)
\end{equation}
mit $\rank\left(\frac{\partial f(x,u)}{\partial u}\right) = p\ \forall x \in U(x_0), \forall
u \in U(u_0)$ d.h. mit $p$ Eingangsgrößen heißt flach, falls es einen fiktiven
Ausgang $y_f = {(y_{f,1}, \ldots, y_{f,p})}^\text{T}$ gibt für den gilt:
\begin{itemize}
    \item
        Alle Komponenten $y_{f,i}$ ($i \in \{1, \ldots, p\}$) des fiktiven Ausgangs
        lassen sich als Funktion der Zustandsgrößen $x_\nu$ ($\nu \in \{1, \ldots, n\}$)
        und $u_i$ ($i \in \{1, \ldots, p\}$) sowie einer endlichen Zahl von Zeitableitungen
        $u_i^{(k_i)}$ ($k_i \in \{1, \ldots, \alpha_i\}$) ausdrücken, also:        
        \begin{eqnarray}
            y_f &=& \psi(x, u_1, \dot{u}_1, \ldots, u_1^{(\alpha_1)}, \ldots
                u_p, \dot{u}_p, \ldots, u_p^{(\alpha_p)}) \\
                &=& \psi(x, u, \dot{u}, \ldots, u^{(\alpha)})
        \end{eqnarray}
        mit den Ableitungstupel:
        \begin{equation}
            u^{(k)} = (u_1^{(k)}, \ldots, u_p^{(k)}
        \end{equation}
    \item  
        Die Zustandsgrößen $x_\nu$ ($\nu \in \{1, \ldots, n\}$) und die Eingangsgrößen
        $u_i$ ($i \in \{1, \ldots, p\}$) lassen sich als Funktionen von $y_{f,i}$
        ($i \in \{1, \ldots, p\}$) und einer endlichen Anzahl von deren Zeitableitungen
        $y_{f,i}^{(k_i)}$ ($k_i \in \{1, \ldots, \kappa_i\}$) darstellen, also:
        \begin{eqnarray}
            x &=& \psi_x (y_{f,1}, \ldots, y_{f,1}^{(\kappa_1 - 1)}, \ldots,
                y_{f, p}, \ldots, y_{f, p}^{(\kappa_p -1)}) \\
                &=& \psi_x(y_f, \ldots, y_f^{(\kappa-1)}) \\
            u &=& \psi_u (y_{f,1}, \ldots, y_{f,1}^{(\kappa_1)}, \ldots,
                y_{f, p}, \ldots, y_{f, p}^{(\kappa_p)}) \\
                &=& \psi_u(y_f, \ldots, y_f^{(\kappa)})
        \end{eqnarray} 
    \item Die Komponenten von $y_f$ sind differentiell unabhängig, d.h. sie erfüllen keine
        Differentialgleichung der Form $\phi(y_f, \ldots, y_f^{(\gamma)}) = 0$.
\end{itemize}

\section{Flachheitsbasierter Steuerungsentwurf}
Problem: Bestimmung einer Solltrajektorie $x_s$ und einer Steuerung $u_s$ für einen
Arbeitspunktwechsel (2-Punkt-Randwertproblem).

Für flaches System:
\begin{itemize}
    \item $y_f(0)$ und $y_f(T)$ über $\psi)$ bestimmen
    \item Trajektorienplanung in $y_f$-Koordinaten
    \item Zustandsverlauf und Steuerungsverlauf mit $\psi_x$ und $\psi_u$ bestimmen
    \item Steuerung in $x$-Koordinaten mit $u$
\end{itemize}

\section{Flachheitsbasierter statischer und dynamischer Folgeregler}
\subsection{Transformation auf Brunovsky-Normalform}
\begin{enumerate}
    \item
        \textbf{Bestimmung von $\psi_{\bar{u}}$:}

        Regelung ist zustandslinear, falls:
        \begin{equation}
            y_{f,i}^{(\kappa_i)} = \bar{u}_i\ \ \forall i \in \{1, \ldots, p\}
        \end{equation}

        Differentielle Parametrierung:
        \begin{equation}
            \psi_{\bar{u}}(y_f^{(\kappa)}) = y_f^{(\kappa)} = \bar{u} 
        \end{equation}
    \item
        \textbf{Bestimmung von $\psi_\xi$:}

        Betrachte inverse Koordinatentransformation der Regelung:
        \begin{equation}
            \begin{pmatrix} x \\ \xi \end{pmatrix} =
            \begin{pmatrix} \psi_x(z) \\ \psi_\xi(z) \end{pmatrix} =
            \phi^{-1}(z)        
        \end{equation}

        Es muss gelten:
        \begin{equation}
            \det \frac{\partial \phi^{-1}}{\partial x}(x_0) \neq 0
        \end{equation}
        $\rightarrow \psi_\xi(z)$ entsprechend bestimmen das erfüllt.
    \item
        \textbf{Bestimmung der Reglerdifferentialgleichung:}

        \begin{eqnarray}
            \dot{\xi} &=& \frac{\partial \psi_\xi(z)}{\partial z} \dot{z} \\
                &=& \bar{a}(z, \bar{u}) \\
                &=& \bar{a}(\psi(x, \xi), \bar{u}) \\
                &=& a(x, \xi, \bar{u})
        \end{eqnarray}
        
        Problem: Bestimmung von $\psi(x, \xi)$ durch Inversion von $\psi^{-1}(z)$.
    
        \begin{eqnarray}
            z &=& {(y_{f,1}, \ldots, y_{f,1}^{(\kappa_1 - 1)}, \ldots, y_{f,p}, \ldots,
                    y_{f,p}^{(\kappa_p - 1)})}^\text{T} \\
              &=& \bar{\phi}(x, u, \dot{u}, \ldots, u^{(\mu)})
        \end{eqnarray}
        mit $\mu_i \geq \alpha_i$ ($i \in \{1, \ldots, p\}$).

        Einsetzten in differentielle Parametrierung des Reglerzustandes:
        \begin{equation}
            \xi = \psi_\xi \circ \bar{\phi}(x, u, \dot{u}, \ldots, u^{(\mu)})
                = \tilde{\phi}(x, u, \dot{u}, \ldots, u^{(\mu)})
        \end{equation}

        Auflösen nach $u$ und dessen Zeitableitungen (meist einfacher als
        ${(\phi^{-1})}^{-1}$ direkt zu berechnen):
        \begin{equation}
            (u, \dot{u}, \ldots, u^{(\mu)}) = \tilde{\psi}^{-1}(x, \xi)
        \end{equation}

        Einsetzten in $\bar{\psi}$:
        \begin{equation}
            \psi(x, \xi) = \bar{\psi}(x, \tilde{\psi}^{-1}(x, \xi))
        \end{equation}
    \item 
        \textbf{Bestimmung der Reglerausgangsgleichung:}

        Einsetzten von $y_f^{(\kappa)} = \bar{u}$ in $\psi_u$ führt zu:
        \begin{equation}
            u = \psi_u(z, \bar{u}) = \psi_u(\psi(x, \xi), \bar{u}) = b(x, \xi, \bar{u}) 
        \end{equation}
\end{enumerate}

\subsection{Folgereglerentwurf in den Brunovsky-Koordinaten}
Die Regler für die Fehlerdynamiken der einzelnen Ausgänge können einzeln über Polvorgabe
entworfen werden.

\section{Nichtlineare Folgebeobachter}
\subsection{Ansatz}
Simulationsterm und Korrekturterm:
\begin{equation}
    \dot{\hat{x}} = f(\hat{x}, u) + L(t) (y- h(\hat{x}) 
\end{equation}
mit $\hat{x}(0) = \hat{x}_0 \in \mathbb{R}^n$.

Bestimmung von $L(t)$ mittels Jacobi-Linearisierung der Fehlerdifferentialgleichung 
entlang der Solltrajektorie $x_s(t)$ für $u_s(t)$.

Satz von Taylor:
\begin{eqnarray}
    f(x,u) &\approx& f(x_s, u_s) + \frac{\partial f}{\partial x}(x_s, u_s) (x -x_s)
                + \frac{\partial f}{\partial u}(u-u_s) \\
            &=& \dot{x}_s(t) + A(t) \Delta x + B(t) \Delta u \\
    f(\hat{x}, u) &\approx& \dot{x}_s(t) + A(t) \Delta \hat{x} + B(t) \Delta u \\
    h(x) &\approx& h(x_s) + \frac{\partial h}{\partial x}(x_s) (x-x_s) \\
            &=& y_s(t) + C(t) \Delta x \\
    h(\hat{x}) &=& y_s(t) + C(t) \Delta \hat{x}
\end{eqnarray}

Einsetzten in die Fehlerdifferentialgleichung liefert:
\begin{equation}
    \Delta \dot{x} - \Delta \dot{\hat{x}} \approx
        (A(t) - L(t) C(t)) (\Delta x - \Delta \hat{x})
\end{equation}

$\Rightarrow (C(t), A(t))$ muss beobachtbar sein, hängt von der Solltrajektorie ab.

\subsubsection{$M_A$-Operator}
Es gilt:
\begin{equation}
    M_A c^\text{T} (t) = c^\text{T} (t) A(t) + \dot{c}^\text{T} (t)
\end{equation}

\subsubsection{$N_A$-Operator}
Es gilt:
\begin{equation}
    N_A s^1(t) = A(t) s^1(t) - \dot{s}^1(t)
\end{equation}

\subsubsection{Beobachtbarkeit}
Zeitvariante Beobachtbarkeitsmatrix:
\begin{equation}
    Q_B(t) =
    \begin{pmatrix}
        c^\text{T}(t) \\ M_A c^\text{T}(t) \\ \vdots \\ M_A^{n-1} c^\text{T} (t)
    \end{pmatrix}
\end{equation}
Falls $\det Q_B(t) \neq 0\ \forall t \in [0, T]$ dann ist das System $(c^\text{T}, A(t)$
gleichmäßig beobachtbar für alle $t \in [0, T]$.

\subsection{Zeitvariante Ackermann-Formel}
\begin{enumerate}
    \item
        Festlegung des Eigenwertvorgabepolynoms $\tilde{\bar{N}}(s)$:
        \begin{equation}
            \tilde{\bar{N}}(s) = \prod_{i=1}^n (s-\lambda_{B,i})
                = s^n + \tilde{\bar{a}}_{n-1} s^{n-1} + \cdots + \tilde{\bar{a}}_1 s +
                    \tilde{\bar{a}}_0
        \end{equation}
    \item
        Bestimmung des Startvektors $s^1(t)$:
        \begin{equation}
            s^1(t) = Q_B^{-1} (t) e_n \bar{c}_n (t)\ \ \bar{c}_n(t) \neq 0, t \geq t_0
        \end{equation}
    \item Zeitvariante Ackermann-Formel:
        \begin{equation}
            T^{-1}(t) = \begin{pmatrix}
                s^1(t) & N_A s^1(t) & \cdots & N_A^{n-1} s^1(t)
            \end{pmatrix}
        \end{equation}
        
        Es gilt:
        \begin{equation}
            l(t) = \frac{1}{\bar{c}_n(t)} (N_A^n s^1(t) + 
                \tilde{\bar{a}}_{n-1} N_A^{n-1} s^1(t) + \cdots +
                \tilde{\bar{a}}_1 N_A s^1(t) + \tilde{\bar{a}}_0 s^1(t)
        \end{equation}
\end{enumerate}


\subsubsection{Lyapunov-Transformation}
Eine lineare, zeitvariante Transformation $z=T(t) x$ mit $\norm{T(t)} < c_1, c_1 > 0$ und
$\norm{T^1(t)} < c_2, c_2 > 0, t \geq t_0 \geq 0$ heißt Lyapunov Transformation.

Exponentielle Stabilität wird durch eine Lyapunov-Transformation erhalten.

\subsection{Nichtlineare Mehrgrößen-Folgebeobachter}
Annahme: Beobachtbarkeitsindizes $\mu_i$ ($i \in \{1, \ldots, m\}$) seien bekannt
und zeitinvariant.

\begin{enumerate}
    \item
        Beobachtbarkeitsanalyse:
        \begin{equation}
            \tilde{Q}_B (t) =
                \begin{pmatrix}
                    c_1^\text{T}(t) \\
                    \vdots
                    M_A^{\mu_1 -1} c_1^\text{T}(t)
                    \vdots
                    c_m^\text{T}(t) \\
                    \vdots
                    M_A^{\mu_m -1} c_m^\text{T}(t)
                \end{pmatrix}
        \end{equation}
        mit $c_i^\text{T}$ Zeilen von $C(t)$. 
        Für Beobachtbarkeit muss $\det \tilde{Q}_B (t) \neq 0\ \forall t \in [0, T]$.
    \item
        Festlegung des Eigenwertvorgabepolynoms $\tilde{\bar{N}}(\lambda)$:
        \begin{equation}
            \tilde{\bar{N}}(\lambda) = \prod_{i=1}^m \prod_{j=1}^{\mu_i}
            (\lambda - \lambda^i_{B,j}) =
            \prod_{i=1}^m (\lambda^{\mu_i} + \gamma_{\mu_i -1}^i \lambda^{\mu_i-1}
            + \cdots + \gamma_1^i \lambda + \gamma_0^i)
        \end{equation}
    \item
        Bestimmung der Startvektoren:
        \begin{equation}
            \begin{pmatrix}
                s^1(t)& s^2(t)& \ldots& s^m(t)
            \end{pmatrix} =
            \tilde{Q}_B^{-1} (t) {(C^*(t))}^\text{T}
        \end{equation} 
        mit ($c_j^{i*}(t)$ beliebig):
        \begin{equation}
            C^*(t) = \begin{pmatrix}
                0 & \cdots & c_1^{1*}(t) & \cdots & 0 & \cdots & c_1^{m*}(t) \\
                \vdots & \ddots & \vdots & \ddots & \vdots & \ddots & \vdots \\
                0 & \cdots & c_m^{1*}(t) & \cdots & 0 & \cdots & c_m^{m*}(t)
            \end{pmatrix}
        \end{equation}

        dabei muss
        \begin{equation}
            0 \neq \det \bar{C}^* (t) = \det \begin{pmatrix}
                c_1^{1*}(t) & \cdots & c_1^{m*}(t) \\
                \vdots & \ddots & \vdots \\
                c_m^{1*}(t) & \cdots & c_m^{m*}(t)
            \end{pmatrix}
        \end{equation}
    \item Zeitvariante-Ackermann-Formel:
        \begin{equation}
            L(t) = \left( \sum_{k=1}^m \sum_{j=0}^{\mu_k -1}
                \gamma_j^k N_A^j s^k(t) + N_A^{\mu_k -1} s^k(t) + \cdots \right) 
                \bar{C}^{*-1}(t)
        \end{equation}
\end{enumerate}
