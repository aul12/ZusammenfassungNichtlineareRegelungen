\chapter{Stabilitätskriterien für die Ruhelage nichtlinearer Systeme}
\section{Direke Methode von Lyapunov}
\subsection{Lyapunov-Testfunktion}
Eine Funktion $V: \mathbb{R}^n \to \mathbb{R}$ für die in $U(0)$ gilt
\begin{itemize}
    \item Stetig Differenzierbar
    \item $V(0)=0$
    \item $V(x) > 0\ \forall x \in U(0)\setminus\{0\}$
\end{itemize}
heißt Lyapunov Testfunktion.

\subsubsection{Typische Testfunktionen}
\begin{itemize}
    \item
        Quadratische Formen ($V(x) = x^\text{T} P x$, mit $P$ positiv definit 
        (Test über Eigenwerte oder Sylvester-Kriterium (NW-Determinanten)))
    \item
        Funktionen die in $0$ ein (lokales) minimum haben.
\end{itemize}

\subsection{Direkte Methode von Lyapunov}
Für ein System mit Ruhelage $x_R=0$ und einer Lyapunov-Testfunktion $V(x)$ gilt:
\begin{itemize}
    \item
        Falls $\dot{V}(0) = 0$ und
        \begin{equation}
            \dot{V}(x) = \frac{\partial V(x)}{\partial x} \dot{x}
                = \frac{\partial V(x)}{\partial x} f(x) \leq 0 
        \end{equation}
        für $x \in U(0)$, d.h. $\dot{V}(x)$ ist negativ semidefinit in $U(0)$,
        dann ist die Ruhelage $x_R=0$ stabil.
    \item
        Falls $\dot{V}(0) = 0$ und
        \begin{equation}
            \dot{V}(x) = \frac{\partial V(x)}{\partial x} \dot{x}
                = \frac{\partial V(x)}{\partial x} f(x) < 0 
        \end{equation}
        für $x \in U(0)\setminus \{0\}$, d.h. $\dot{V}(x)$ ist negativ definit in $U(0)$,
        dann ist die Ruhelage $x_R=0$ asymptotisch stabil.
\end{itemize}
$V(x)$ wird dann Lyapunovfunktion des Systems genannt.

\subsection{Abschätzung des Einzugsbereichs}
Sei $f$ ein System mit Ruhelage $x_R=0$ und $V(x)$ eine Lyapunov-Testfunktion. Eine
Menge $S$ mit $\dot{V}(0) = 0$ und $\dot{V}(x) < 0, x \neq 0$ kann zur Abschätzung des
Einzugsbereichs genutzt werden:

Sei $B$ ein beschränktes Gebiet in $S$ mit:
\begin{itemize}
    \item $x_R \in B$
    \item $V(x) < c, c>0, x \neq 0$
    \item Der Rand wird durch $V(x)=c$ gebildet
\end{itemize}
dann ist die Ruhelage asymptotisch stabil und $B$ gehört zu ihrem Einzugsbereich.

\subsection{Global asymptotisch stabile Ruhelage}
Sei $f$ ein System mit Ruhelage $x_R=0$, $V(x)$ eine Lyapunov-Funktion im
\textbf{gesamten Zustandsraum} mit:
\begin{itemize}
    \item
        $\dot{V}(0) = 0$
    \item
        $\dot{V}(x) < 0\ \forall x \neq 0$
\end{itemize}
darüber hinaus sei $V(x)$ \textbf{radial unbeschränkt}, d.h.
\begin{equation}
    \norm{x} \to \infty \Rightarrow V(x) \to \infty 
\end{equation}
dann ist $x_R$ global asymptotisch stabil.

\subsection{Exponentielle Stabilität der Ruhelage}
Sei $f$ ein System mit Ruhelage $x_R=0$, $V(x)$ eine Lyapunov-Funktion in $U(0)$ mit:
\begin{itemize}
    \item
        $\alpha \norm{x}^2 \leq V(x) \leq \beta \norm{x}^2\ \alpha, \beta > 0$
    \item
        $\dot{V}(x) < -\gamma \norm{x}^2\ \gamma > 0$
\end{itemize}
Dann ist $x_R$ exponentiell stabil mit $c=\frac{\gamma}{2 \beta}$ als Rate der exponentiellen Konvergenz.

\section{Invarianzprinzip von LaSalle (asymptotische Stabilität mit negativ semidefiniter Zeitableitung)}
Sei $f$ ein System mit Ruhelage $x_R=0$, und $S$ eine Punktmenge die $x_R$ enthält und in
der eine Lyapunov-Funktion $V(x)$ die Eigenschaften $\dot{V}(0) = 0$ und 
$\dot{V}(x) \leq 0$ besitzt.
Sei $B$ eine Teilmenge von $S$ mit $x_R \in B$ und $V(x) < c$ gilt, mit Rand
$V(x) = c$ mit $c > 0$ (vgl. Einzugsbereich).
Wenn in $B$ die Punktmenge auf der $\dot{V}(x) = 0$ ist außer $x_R=0$ keine Trajektorien
des Systems enthält, dann ist $x_R$ asymptotisch stabil und $B$ gehört zum Einzugsbereich.

Ist $V(x)$ radial unbeschränkt und gelten die Aussagen im gesamten Zustandsraum, dann ist
die Ruhelage global asymptotisch stabil.

\section{Methode der ersten Näherung}
Sei $f$ ein System mit Ruhelage $x_R=0$ und $f$ stetig differenzierbar in $U(0)$.
$A$ ist die Jacobi-Matrix von $f$ im Ursprung, also $A = \frac{\partial f}{\partial x}(0)$.

Dann gilt:
\begin{itemize}
    \item
        Liegen alle Eigenwerte links der Imaginärachse, dann ist die RL asymptotisch stabil
    \item
        Liegt mindestens einer der Eigenwerte rechts der Imaginärachse, dann ist die
        RL instabil
    \item Liegt mindestens einer der Eigenwerte auf der Imaginärachse und die anderen
        links, so hat man keine Stabilitätsausage (kritischer Fall)
\end{itemize}
