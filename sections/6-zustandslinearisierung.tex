\chapter{Exakte Zustandslinearisierung}
\section{Transformation auch nichtlineare Regelungsnormalform}
Ansatz: Bestimmung eines fiktiven Ausgangs $y_f$ ohne Nulldynamik, d.h. $r_f = n$ zur
exakten Zustandslinearisierung.

\subsection{Zusammenhang zwischen Lie-Ableitungen und Lie-Klammern}
Es gilt:
\begin{equation}
    L_{ad_f g(x)} h(x) = L_f L_g h(x) - L_g L_f h(x)
\end{equation}
mit
\begin{equation}
    ad_f g(x) = [f(x), g(x)]
\end{equation}
und
\begin{equation}
    ad_f^k g(x) = ad_f (ad_f^{k-1} g(x))
\end{equation}
sowie
\begin{equation}
    ad_f^0 g(x) = g(x)
\end{equation}

\subsection{Anforderungen an den Ausgang}
Es muss gelten:
\begin{eqnarray}
    L_g h_f(x) &=& 0 \\
    L_{ad_f g} h_f(x) &=& 0 \\
    L_{ad_f^{n-2} g} h_f(x) &=& 0
\end{eqnarray}
Formuliert als Frobenius-PDgl.:
\begin{equation}
    \frac{\partial h_f(x)}{\partial x} \left(g(x), ad_f g(x), \ldots, ad_f^{n-2} g(x)\right) = 0
\end{equation}

Außerdem soll gelten:
\begin{equation}
    L_{ad_f^{n-1} g} h_f(x) = {(-1)}^{n-1} b(x)
\end{equation}

\subsection{Existenz der Lösung}
Die Lösung existiert genau dann falls:
\begin{enumerate}
    \item 
        $\Delta = \text{span}(g, ad_f g, \ldots, ad_f^{n-2} g)$ reguläre Distribution
        bei $x_0$ mit $\dim(\Delta(x)) = n-1$ in $U(x_0)$.

        Bzw. da $b(x)$ nicht verschwinden soll:
        \begin{equation}
            \det(g(x_0), ad_f g(x_0), \ldots, ad_f^{n-2} g(x_0), ad_f^{n-1} g(x_0)) \neq 0
        \end{equation}
        äquivalent dazu ($A, b$ aus Jakobi-Linearisierung):
        \begin{equation}
            \det(b, Ab, \ldots, A^{n-1} b) \neq 0
        \end{equation}
    \item
        $\Delta$ muss involutiv sein, d.h 
        \begin{equation}
            [ad_f^i g(x), ad_f^j g(x)
                = \sum_{k=0}^{n-2} \alpha_k^{ij} (x) ad_f^k g(x)
        \end{equation}
        mit $\alpha_k^{ij} \in C^\infty (U(x_0))$ und $i, j \in \{0, \ldots, n-2\}$
\end{enumerate}

\subsection{Zusammenhang Steuerbarkeit und exakte Zustandslinearisierung}
Es gilt: 

$\exists y_f \text{ mit } r_f = n \text{ bei } x_0 \Leftrightarrow$

exakte Zustandslinearisierbarkeit in steuerbares, lineares System durch statische 
Zustandsrückführung

\section{Stabilisierung mittels nichtlinearer Ackermann-Formel}
\begin{enumerate}
    \item
        Überprüfen auf exakte Zustandslinearisierung ($\Leftrightarrow$
        fiktiver Ausgang $y_f$ mit $r=n$ finden.
    \item
        Transformation auf nichtlineare Regelungsnormalform:
        \begin{eqnarray}
            \dot{z}_1 &=& z_2 \\
            \dot{z}_{n-1} &=& z_n \\
            \dot{z}_n &=& L_f^n h_f(x) + L_g L_f^{n-1} h_f(x)
        \end{eqnarray}
    \item
        Kompensation der nichtlinearitäten:
        \begin{equation}
            u = \frac{1}{L_g L_f^{n-1}} (-L_f^n h_f(x) + \bar{u})
        \end{equation}
    \item
        Stabilisierung durch Eigenwertvorgaben in Brunovsky-Normalform.
        
        Lineare Zustandsrückführung:
        \begin{equation}
            \bar{u} = -\tilde{a}_{n-1} z_n - \ldots - \tilde{a}_1 z_2 - \tilde{a}_0 z_1
        \end{equation}

        Führt auf geregeltes System:
        \begin{equation}
            \dot{z} = \tilde{A} z
        \end{equation}

        mit:
        \begin{equation}
            \tilde{A} =
                \begin{pmatrix}
                    0 & 1 & 0 & \cdots & 0 \\
                    0 & 0 & 1 & \cdots & 0 \\
                    \vdots & & & \ddots & 1 \\
                    -\tilde{a}_0 & & \cdots & & -\tilde{a}_{n-1}
                \end{pmatrix}
        \end{equation}

        Stabilitätsüberprüfung über EW/Hurwitz-Kriterium.
    \item
        Regler in Originalkoordinaten:
        \begin{equation}
            u = \frac{1}{L_g L_f^{n-1}} (-L_f^n h_f(x) 
            - \tilde{a}_{n-1} L_f^{n-1} h_f(x) - \cdots - \tilde{a}_1 L_f h_f(x)
            - \tilde{a}_0 h_f(x))
        \end{equation}
\end{enumerate}

Bemerkungen:
\begin{itemize}
    \item
        Einzugsbereich meist größer als bei Jakobi-Linearisierung
    \item
        Falls $\phi$ global, dann ist Ruhelage global asymptotisch stabil.
    \item 
        Dynamik in $x$-Koordinaten nicht linear!
\end{itemize}
