\chapter{Übertragungs- und Zustandslinearität}
\section{Ein-/Ausgangslinearität}
Ein (Single-Input-Single-Output) System mit:
\begin{eqnarray}
    \dot{x} &=& f(x, u)\ x(0) =x_0 \in \mathbb{R}^n \\
    y &=& h(x)
\end{eqnarray}
mit Eingangsgröße $u(t) \in \mathbb{R}$ und Ausgangsgröße $y(t) \in \mathbb{R}$ ist
ein-/ausgangslinear (EA-Linear) bzw. übertragungslinear wenn seine 
Ein-/Ausgangsdifferentialgleichung linear ist. D.h.
\begin{equation}
    y^{(m)} + a_{m-1} \cdot y^{(m-1)} + \cdots + a_0 y =
    b_0 u + \cdots b_q \cdot u^{(q)}
\end{equation}
mit $m \leq n$ und $q < n$ (also ohne Durchgriff). Außerdem wird Zeitinvarianz
vorausgesetzt ($a_k$ und $b_k$ sind konstant).

Für MIMO-Systeme muss jeder Ausgang diese Bedingung erfüllen. Für $x_0 \neq 0$ verletzt
das System die strenge Linearitätsanforderung wird jedoch ebenfalls oftmals als linear
bezeichnet.

\section{Eingangs-/Zustandslinearität}
Ein dynamisches System
\begin{equation}
    \dot{x} = f(x, u)\ x_0 \in \mathbb{R}^n
\end{equation}
mit Eingangsgröße $u \in \mathbb{R}$ ist (eingangs-) zustandlinear wenn $f$ eine lineare
Vektorfunktion ist, d.h. falls
\begin{equation}
    f(x, u) = A x + b u
\end{equation}

Im MIMO-Fall ist $b$ ein Matrix. Zustandslinearität ist keine Systemeigenschaft, da
sie von der Wahl der Zustandsgrößen abhängt.

\section{Zusammenhang zwischen Übertragungs und Zustandlinearität}
Die beiden Eigenschaften sind unabhängig:
\begin{itemize}
    \item System mit nichtlinearem $y$ ist Zustandslinear aber nicht EA-Linear
    \item System mit nicht messbarem, nichtlinearen Zustand ist EA-Linear
        aber nicht Zustandslinear
\end{itemize}

