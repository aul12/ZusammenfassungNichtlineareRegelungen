\chapter{Exakte Ein-/Ausgangslinearisierung}
\section{Bestimmung des exakt E/A-linearisierenden Stellgesetzes}
\subsection{Problemstellung}
Für ein eingangsaffines nichtlineares SISO-System
\begin{eqnarray}
    \dot{x} &=& f(x) + g(x) u \\
    y &=& h(x)
\end{eqnarray}
soll eine nichtlineare Zustandsrückführung
\begin{equation}
    u = \alpha(x) + \beta(x) + \bar{u}
\end{equation} 
bestimmt werden, so dass die Verknüpfung der beiden Systeme EA-Linear ist.

Es wird angenommen, dass $f, g, h$ hinreichend oft stetig differenzierbar sind.

\subsection{Lie-Ableitung}
Es gilt:
\begin{equation}
    L_f^i h(x) = \frac{\partial L_f^{i-1} h(x)}{\partial x} f(x)
\end{equation}
außerdem gilt:
\begin{equation}
    L_f^0 h(x) = h(x)
\end{equation}

\subsection{Differenzgrad}
Die Größe $r$ heißt Differenzgrad der Ausgangsgröße $y=h(x)$ bei $x=x_0$ (Differenzgrad
kann variieren) wenn gilt:
\begin{eqnarray}
    L_g L_f^i h(x) &=& 0\ \text{in } U(x_0), i \in \{0, \ldots r-2\} \\
    L_g L_f^{r-1} h(x_0) &\neq& 0
\end{eqnarray}

\subsubsection{Bei linearen Systemen}
Bei linearen Systemen ist $r$ gleich dem Differenzgrad von Nenner- und Zählerpolynom
der Übertragungsfunktion.

\subsection{Herleitung}
Idee: Kompensation der nichtlinearitäten:
\begin{equation}
    y^{(r)} = L_f^r h(x) + L_g L_f^{r-1} h(x) u \stackrel{!}{=} \bar{u}
\end{equation}
da $L_g L_f^{r-1} h(x) \neq 0$ kann man nach $u$ auflösen:
\begin{equation} \label{eq:u}
    u = \frac{1}{L_g L_f^{r-1} h(x)} \left(-L_f^r h(x) + \bar{u} \right)
      = \alpha(x) + \beta(x) \bar{u}
\end{equation}
Vorgabe der linearen Dynamik, z.B. über Polvorgabe ($w$ ist der Regelfehler):
\begin{equation}
    y^{(r)} = - \tilde{a}_{r-1} y^{(r-1)} - \cdots - \tilde{a}_1 \dot{y} - \tilde{a}_0 y +
        \tilde{a}_0 w
\end{equation}
außerdem gilt:
\begin{equation}
    y^{(i)} = L_f^i h(x)
\end{equation}
einsetzten in Gleichung \ref{eq:u} liefert:
\begin{eqnarray}
    u &=& \frac{1}{L_g L_f^{r-1} h(x)} \\
        &&\left(-L_f^r h(x) - \tilde{a}_{r-1} L_f^{r-1} h(x) - \ldots - \tilde{a}_1 L_f h(x) - \tilde{a}_0 h(x) + \tilde{a}_0 w \right) \\
      &=& \tilde{\alpha}(x) + \tilde{\beta}(x) \bar{u}
\end{eqnarray}

Für $r<n$ existiert ein nichtbeobachtbares Teilsystem dessen Stabilität untersucht werden
muss.

\section{Byrnes-Isidori-Normalform}
\subsection{Koordinatentransformation}
Eine Abbildung $z = \phi(x): D \subset \mathbb{R}^n \to G \subset \mathbb{R}^n$ zwischen
den Gebieten $D$ und $G$ ist eine lokale Koordinatentransformation (Diffeomorphismus),
wenn:
\begin{itemize}
    \item
        $\phi(x)$ in $U(x_0)$ invertierbar ist, d.h. es existiert ein $\phi^{-1}(z)$ mit
        $\phi^{-1}(\phi(x)) = x\ \forall x \in U(x_0) \subset D$.
    \item
        $\phi(x)$ und $\phi^{-1}(z)$ müssen in $U(x_0)$ bzw. $U(\phi(x_0))$ glatt
        sein. 
\end{itemize}

\subsection{Implizites Funktionentheorem}
$\phi(x)$ sei eine in $U(x_0)$ $k$-fach stetig differenzierbare Funktion, also
$\phi \in C^k(U(x_0))$, dann gilt auch $\phi^{-1} \in C^k(U(\phi(x_0)))$ falls
\begin{equation}
    \det \frac{\partial \phi}{\partial x}(x_0) \neq 0
\end{equation}

\subsection{Bestimmung der Koordinatentransformation}
\begin{enumerate}
    \item
        $r$ Koordinaten für $\xi$ Teil bestimmen (in $U(x_0)$):
        \begin{eqnarray}
            \xi_1 &=& y = h(x) \\
            \xi_2 &=& \dot{y} = L_f h(x) \\
            &\vdots& \\
            \xi_r &=& y^{(r-1)} = L_f^{r-1} h(x)
        \end{eqnarray} 
        die Transformation wird mit $\phi_\xi$ bezeichnet.
    \item
        $n-r$ Koordinaten für $\eta$ Teil bestimmen (interne Dynamik) mit
        $\eta = {(\eta_1, \cdots \eta_{n-r})}^\text{T} = \phi_\eta(x)$ und der
        Gesamttransformation $\phi(x) = {(\phi_\xi(x), \phi_\eta(x))}^\text{T}$
        muss gelten:
        \begin{equation}
            \det \frac{\partial \phi}{\partial x}(x_0) \neq 0
        \end{equation}
\end{enumerate}

\subsection{Transformation des Systems}
$\xi$-Teilsystem (vgl. Regelungsnormalform):
\begin{eqnarray}
    \dot{\xi}_1 &=& \dot{y} = \xi_2 \\
    \dot{\xi}_2 &=& \ddot{y} = \xi_3 \\
        &\vdots& \\
    \dot{\xi}_{r-1} &=& \xi_r \\
    \dot{\xi}_r &=& L_f^r h \circ \phi^{-1}(\xi, \eta) + L_g L_f^{r-1} h \circ \phi^{-1}(\xi, \eta) u \\
        &=& a(\xi, \eta) + b(\xi, \eta) u
\end{eqnarray}

$\eta$-Teilsystem:
\begin{eqnarray}
    \dot{\eta} &=& \frac{\partial \phi_\eta(x)}{\partial x} \dot{x} \\
        &=& \frac{\partial \phi_\eta(x)}{\partial x} f \circ \phi^{-1}(\xi, \eta) +
        \frac{\partial \phi_\eta(x)}{\partial x} g \circ \phi^{-1}(\xi, \eta) u \\
        &=& q(\xi, \eta) + p(\xi, \eta) u
\end{eqnarray}

Ausgang:
\begin{equation}
    y = \xi_1
\end{equation}

\section{Nulldynamik}
\subsection{Stabilitätsuntersuchung des $\eta$-Teilsystems}
\subsubsection{Fall $r=n$}
Für $r=n$ existiert keine interne Dynamik, die Ruhelage $x_0$ kann immer durch EA-Linearisierung stabilisiert werden.

Das System wird durch das Stellgesetz in $U(x_0)$ EA und Zustandslinearisiert (totale Linearisierung).

\subsubsection{Fall $r < n$}
Ansatz: Methode der ersten Näherung (problematisch im kritischen Fall). Außerdem kann
die Ruhelage auch in $(\xi, \eta)$-Koordinaten in den Ursprung transformiert werden.

Das (geregelte) $\xi$-Teilsystem ist gegeben durch:
\begin{equation}
    \dot{\xi} = J \xi + e_r (a(\xi, \eta) + b(\xi, \eta) u)
\end{equation}
mit
\begin{equation}
    J =
    \begin{pmatrix}
        0 & 1 & \cdots & 0 \\
        \vdots & & \ddots & \\
        0 & \cdots & 0 & 1 \\
        0 & \cdots & 0 & 0
    \end{pmatrix}
\end{equation}
und (durch Regler):
\begin{equation}
    a(\xi, \eta) + b(\xi, \eta) u = - \tilde{a}_{r-1} \xi_r - \cdots - \tilde{a}_0 \xi_1
\end{equation}

Das (geregelte) $\eta$-Teilsystem ist gegeben durch:
\begin{equation}
    \dot{\eta} = q(\xi, \eta) + p(\xi, \eta) 
    \frac{-a(\xi, \eta) - \tilde{a}_{r-1} \xi_r - \cdots - \tilde{a}_0 \xi_1}{b(\xi, \eta)}
    = \tilde{q}(\xi, \eta)
\end{equation}

Für das Gesamtsystem folgt mit dem Satz von Taylor (und der Ruhelage):
\begin{equation}
    \begin{pmatrix} \dot{\xi} \\ \dot{\eta} \end{pmatrix} =
    \begin{pmatrix} \tilde{J} & 0 \\ P & Q \end{pmatrix}
    \begin{pmatrix} \xi \\ \eta \end{pmatrix}
\end{equation}
Wobei $\tilde{J}$ eine Regelungsnormalformmatrix mit $\tilde{a}_k$ ist.

Die Stabilität hängt rein von den Eigenwerten der $Q$ Matrix und der Regelung ab.

\subsection{Definition Nulldynamik}
Die für $y(t) \equiv 0$ noch im System vorhandene Dynamik wird Nulldynamik genannt
und kann durch:
\begin{equation}
    \dot{\eta} = \tilde{q}(0, \eta)\ \ \eta(0) = \eta_0 \in \mathbb{R}^{n-r}
\end{equation}
bestimmt werden.

\subsection{Starke Übertragungsblockierung}
Die Eigenschwingungen der Nulldynamik sind die einzigen Signale, die stark Übertragungsblockiert werden können.

\subsection{Stabilität der EA-linearisierten Regelung}
Exakt EA-linearisierte Regelung hat asymptotisch stabile Ruhelage $x_0 = 0$, wenn die
Ruhelage $\eta_0 = 0$ der Nulldynamik asymptotisch stabil ist und $\tilde{J}$ eine Hurwitz-Matrix
ist.

\subsection{Minimalphasig nichtlineare Systeme}
Ein nichtlineares System heißt minimalphasig bei $x_0$ wenn seine Nulldynamik dort eine
asymptotisch stabile Ruhelage besitzt.

\section{Frobenius-Theorem}
\subsection{Distributionen}
Betrachte $k$ glatte Vektorfelder $f_1(x), \ldots f_k(x) \in \mathbb{R}^n$ mit 
$x \in U \subset \mathbb{R}^n$ einer offenen Umgebung. Dann wird durch
\begin{equation}
    \Delta(x) = \text{span}\{f_1(x), \ldots f_k(x)\}
\end{equation}
Punktweise ein Unterraum definiert. Diese Abbildung wird Distribution genannt.

\subsubsection{Reguläre Distributionen}
Eine reguläre Distributionn hat konstante Dimension in einer Umgebung von $x_0$.

\subsection{Frobenius-Theorem}
Eine reguläre Distribution $\Delta$ in $U$ ist genau dann vollständig integrierbar, wenn
sie involutiv ist.

\subsection{Involutitivtät einer Distribution}
Bei involutiven Distributionen führt die Lie-Klammer-Bildung punktweise zu keinen neuen
linear unabhängigen Vektoren.

\subsubsection{Lie-Klammern}
Es gilt:
\begin{equation}
    [\tau_1(x), \tau_2(x)] = \frac{\partial \tau_2(x)}{\partial x} \tau_1(x) - \frac{\partial \tau_1(x)}{\partial x} \tau_2(x)
\end{equation}

\subsubsection{Kriterium}
\begin{equation}
    \rank(f_1(x), \ldots, f_d(x), [f_i(x), \ldots, f_j(x)]) = d\ \ \forall x \in U, i, j \in \{1, \ldots, d\}
\end{equation}

\section{Eingangsnormalisierende Byrnes-Isidori-Normalform}
Die interne Dynamik
\begin{equation}
    \dot{\eta} = q(\xi, \eta) + p(\xi, \eta) u
\end{equation}
soll nicht direkt vom Eingang abhängen, also $p(\xi, \eta) \equiv 0$.

Das heißt:
\begin{equation}
    \frac{\partial \phi_\eta(x)}{\partial x}g(x) \stackrel{!}{=} 0
\end{equation}

da $\Delta = \text{span}(g)$ Dimension $1$ hat ist Distribution involutiv. D.h. solch
ein $\phi_\eta$ existiert.

\section{Steuerungsentwurf mittels Byrnes-Isidori-Normalform}
\subsection{Inverses System für $r<n$}
Interne Dynamik (muss in $S^{-1}$) mit bestimmt werden, z.B. über numerische Integration,
dafür stabil $\Leftrightarrow$ Minimalphasig):
\begin{equation}
    \dot{\eta}_s = q(y_s(t), \ldots, y_s^{(r-1)}(t), \eta_s(t))
\end{equation}
mit $\eta_s(0) = \eta_{s,0} \in \mathbb{R}^{n-r}$.

Stellsignal:
\begin{eqnarray}
    u_s(t) &=& \psi_y(y_s(t), \dot{y}_s(t), \ldots, y_s^{(r)}(t), \eta_s(t)) \\
        &=& \frac{1}{b(\xi_s, \eta_s)} (-a(\xi_s, \eta_s) + y_s^{(r)})
\end{eqnarray}

\subsection{Inverses System für $r=n$}
Keine interne Dynamik (flaches System, keine DGL):
\begin{eqnarray}
    u_s(t) &=& \psi_y(y_s(t), \dot{y}_s(t), \ldots, y_s^{(r)}(t)) \\
        &=& \frac{1}{b(\xi_s)} (-a(\xi_s) + y_s^{(r)})
\end{eqnarray}

entsprechend kann auch die Zustandstrajektorie bestimmt werden:
\begin{equation}
    x_s = \psi^{-1}(\xi_s) = \psi_x(y_s, \ldots, y_s^{(n-1)}
\end{equation}

\section{Ausgangsfolgeregelung}
Probleme:
\begin{itemize}
    \item Inkonsistente Anfangsbedingung
    \item Externe Dauerstörung und Modellunbestimmtheit
\end{itemize}
Ausgangsfolgefehler $e=y-y_s$ muss ausgeregelt werden.

\subsection{Exakte Linearisierung der Fehlerdynamik}
Exakte Linearisierung:
\begin{equation}
    u = \frac{1}{L_g L_f^{r-1} h(x)} (-L_f^r h(x) + \bar{u})
\end{equation}
und der Vorsteuerung mit Regler:
\begin{equation}
    \bar{u} = y_s^{r}\ - \tilde{a}_{r-1} e^{r-1} - \cdots - \tilde{a}_0 e
\end{equation}

\subsection{Lineare Regelung der Fehlerdynamik}
Idee: Nichtlineare Vorsteuerung mit (einfachem) linearen Regler.

\subsubsection{Vorsteuerung}
Problem: für $S^{-1}$ wird $\phi^{-1}$ benötigt, kann aber im Allgemeinen nicht bestimmt werden.

Lösung: nichtlineare Modellgestütze Vorsteuerung.

Für Stellsignal wird zweites System durch exakte Linearisierung gesteuert, dadurch generiertes Steuersignal wird
für Hauptssystem genutzt.

\subsection{Führungsgrößenglättung}
Es wird ein lineares (und stationär genaues) Filter $r$-Ordnung benötigt um eine 
beliebige Führungsgröße zu glätten damit hinreichend oft stetig differentierbar.

Damit folgt Regelgröße $y$ der Führungsgröße $w$ jedoch nur asymptotisch.

